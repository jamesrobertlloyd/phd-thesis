% ************************** Thesis Abstract *****************************
% Use `abstract' as an option in the document class to print only the titlepage and the abstract.
\begin{abstract}
This thesis makes contributions to a variety of aspects of probabilistic inference.
When performing probabilistic inference, one must first represent one's beliefs with a probability distribution.
Specifying the details of a probability distribution can be a difficult task in many situations, but when expressing beliefs about complex data structures it may not even be apparent what functional form such a distribution should take.
This thesis starts by demonstrating how representation theorems due to Aldous, Hoover and Kallenberg can be used to specify appropriate models for data in the form of networks.
These theorems are then extended in order to reveal appropriate probability distributions for arbitrary relational data or databases.

A simpler data structure to specify probability distributions for is that of functions; a great many probability distributions for functions have been used for centuries.
We demonstrate that many of these distributions can be expressed in a common language of Gaussian process kernels constructed from a few base elements and operators.
The structure of this language allows for the effective automatic construction of probabilistic models for functions.
Furthermore, the formal mathematical language of kernels can be mapped neatly onto natural language allowing for automatic descriptions of the automatically constructed models.

By further automating the construction of statistical models, the need to be able to effectively check or criticise these models becomes greater.
This thesis demonstrates how kernel two sample tests can be used to demonstrate where a probabilistic model most disagrees with data allowing for targeted improvements to the model.
In proposing a new method of model criticism this thesis also briefly discusses the philosophy of model criticism within the context of probabilistic inference.
\end{abstract}