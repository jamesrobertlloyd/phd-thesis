\input{common/header.tex}
\inbpdocument

\chapter{Conclusion}
\label{ch:conclusion}

This thesis has made a broad range of contributions to the practice of probabilistic modelling.
In chapter~\ref{ch:networks} we demonstrated how theorems based on symmetry arguments could be used to provide a unifying view of probabilistic models for network data and proposed a new nonparametric model inspired as a literal interpretation of the theorems.
In chapter~\ref{ch:arrays} we extended these theorems to arbitrary relational data, or any data that can be stored in a database.
These results bring further unity and clarity to the ever growing literature of different probabilistic models for such data.
They also highlight the current mismatch between theory and data with regards to sparsity of data.

In chapter~\ref{ch:construction} we demonstrated how a broad range of probabilistic models of functions could be expressed in a common modelling language of Gaussian process kernels.
We demonstrated that this language could be effectively searched to automatically construct appropriate models for particular data and that these models were often highly interpretable.
This claim of interpretability was more rigorously justified in chapter~\ref{ch:description} where we demonstrated that the compositional structure of the language of kernels could be neatly mapped onto compositionally generated natural language descriptions of the models the kernels encode for.

In chapter~\ref{ch:gefcom} we demonstrated the effectiveness of Gaussian processes with compositionally constructed kernels in a data mining competition.
We also demonstrated how the model construction procedure introduced in chapter~\ref{ch:construction} could be used to recreate some of the results of the competition whilst also revealing some limitations with the current approximate inference methods available for Gaussian processes.
Finally, in chapter~\ref{ch:criticism} we considered the problem of criticising a probabilistic model when we have reason to believe that we were either unable to faithfully express our prior information probabilistically or perform inference to a sufficient level of accuracy.
In particular we introduced a new method for model criticism that attempts to find the greatest discrepancies between a probabilistic model and data and thereby target future efforts to improve a particular model.

\outbpdocument{
\bibliographystyle{plainnat}
\bibliography{references.bib}
}
