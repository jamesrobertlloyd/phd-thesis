\input{common/header.tex}
\inbpdocument

\chapter*{Notation}
\label{ch:notation}
\addcontentsline{toc}{chapter}{Notation}

Unless explicitly specified otherwise, I will write probability distributions as imperative program \ie each line indicates how a random variable is to be generated and following each line in order specifies a single sample from the joint distribution.
For example,
\[
  X & \dist \Normal(0,1) \nonumber \\
  Y & \dist \Normal(0,1) \nonumber
\]
would imply that $X$ and $Y$ are normally distributed but also independent and
\[
  X & \dist \Normal(0,1) \nonumber \\
  Y & \dist \Normal(X,1) \nonumber
\]
would imply that $X$ and $Y$ have a covariance of 1.

I will use curly braces $\{\}$ to indicate sets and parentheses $()$ to indicate lists.

\TBD{This will grow as I find other nonstandard notation}

\outbpdocument{
}