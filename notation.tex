\input{common/header.tex}
\inbpdocument

\chapter*{Notation}
\label{ch:notation}
\addcontentsline{toc}{chapter}{Notation}

Distribution notation is often abused when specifying complex joint distributions.
For example
\[
  X & \dist \Normal(0,1) \nonumber \\
  Y & \dist \Normal(0,1) \nonumber
\]
would often be read as $X$ and $Y$ being \emph{independent} standard normals.
But, for the sake of pedantry, this statement has not actually specified anything about the joint distribution; $X=Y$ is consistent with the statements above.
Since this could become confusing in longer statements, I will write probability distributions as imperative programs \ie each line indicates how a random variable is to be generated and following each line in order specifies a single sample from the joint distribution.
This means that each statement is a conditional distribution, conditioned on all previously defined random variables.
To make this clear, I will use the following notation
\[
  X & \probgets \Normal(0,1) \nonumber \\
  Y & \probgets \Normal(0,1) \nonumber
\]
where the symbol $\probgets$ has been chosen since it is similar both to the distribution symbol $\dist$ and programmatic assignment $\gets$\footnotemark{}.


\footnotetext{Ideally I would have used the unicode character `LEFTWARDS WAVE ARROW' (U+219C) but this does not appear to be in any standard \LaTeX\  packages that do not interfere with my current document setup.}

I will use curly braces $\{\}$ to indicate sets and parentheses $()$ to indicate lists.
The index set of lists or sets will occasionally be omitted when clear from context \eg $(U_i)$ might be short hand for $(U_i)_{i\in\Nats}$.

\outbpdocument{
}