\input{common/header.tex}
\inbpdocument

\chapter{Representing probability distributions of exchangeable databases}
\label{ch:arrays}

In chapter~\ref{ch:networks} we demonstrated how to characterise probability distributions over objects such as networks or user--item rating matrices (see remark~\ref{remsep}).
The key properties that allowed us to characterise these distributions were symmetries in the data, or invariances to the way they are stored.
For networks, we assumed that the ordering of the nodes contained no information.
For user--item data we assumed that the order of both users and items were irrelevant to the meaning of the data.

In this chapter we extend these ideas of exchangeability to any data that can be stored in a relational database \ie arbitrary relational data.
We demonstrate that extensions of the Aldous--Hoover theorem to $n$-arrays due to \citet{Kallenberg1999-pj} can be extended to the case of exchangeable databases.
These new theorems reveal a natural parameter space for data stored in relational databases elucidating how to construct appropriate statistical models for such data.

This chapter is largely based on collaborations with Zoubin Ghahramani, Peter Orbanz and Daniel Roy.
In particular, preliminary versions of the technical results presented in this chapter appeared at the Bayesian Nonparametrics Workshop 2012 and a NIPS workshop \citep{Lloyd_undated-iu} but without proof.

\section{Introduction}

Relational databases are an extremely common data structure so it is natural to want to perform statistical tasks with such data \eg predicting unobserved data or identifying latent structure.
In particular, network data is rarely encountered in isolation \eg in a social network one will often have access to information about each user.
To perform a statistical analysis of such data we will typically need to specify a probabilistic model of the data, but it is not immediately clear what an appropriate parameter space for such a model is.
The choice of parameter space is important because it indicates the targets of statistical inference and determines where we can share statistical strength between different aspects of the data.

We demonstrate that the weak assumption of an appropriate form of exchangeability can provide a natural parameter space.
This form of exchangeability is appropriate when the order of the objects underlying a relational database (\eg users and movies in a database of ratings data) is arbitrary or unimportant.
For example, the left hand side of figure~\ref{fig:exchangeable} shows the same network but with differently labeled nodes.
If the labelling is unimportant, then any probabilistic model of such data should assign them the same probability.

\begin{figure}[ht]
\centering
\begin{tabular}{c}
\tiny \input{common/header.tex}
\inbpdocument

\chapter{Representing probability distributions for exchangeable arrays}
\label{ch:arrays}

This chapter\dots

\outbpdocument{
\bibliographystyle{plainnat}
\bibliography{references.bib}
}

\end{tabular}
\caption[Illustration of array exchangeability for network adjacency matrices.]{Left: Networks with equivalent structure but different node labels. Right: Corresponding adjacency matrix representations of these networks}
\label{fig:exchangeable}
\end{figure}

Relational data are typically stored in arrays; the right hand side of figure~\ref{fig:exchangeable} shows the corresponding adjacency matrix representations of the networks on the left.
We demonstrate that exchangeability of the objects underlying a relational database can be expressed in terms of array exchangeability.
Prior work on array exchangeability, both theoretical \citep[e.g.][]{Hoover1979-br, Aldous1981-lg, Hoover1982-ty, Kallenberg1999-pj, Diaconis2007-mi, Aldous2010-iw, Austin2012-jq, Choi2013-th, Wolfe2013-vs} and applied \citep[e.g.][]{Hoff2007-ja,Roy2009-ge,Lloyd2012-sb}, has focused on single exchangaeble arrays.
We show that the representation theorems for single arrays can be used to derive representations for collections of exchangeable arrays \ie exchangeable databases or exchangeable relational data.

\section{Exchangeable databases}

We abstractly define a database following the entity-relationship formalism \citep[e.g.][]{Ullman2002-eo} where the values of attributes are the result of evaluating functions (relations) over a collection of entities / objects.

\newcommand{\bi}{\bm{i}}
\newcommand{\Types}{T}
\newcommand{\Space}{S}
\begin{definition}[types, signatures, relation]
Fix a finite set $\Types$ of \defn{types}.  Define a \defn{signature} to be a finite sequence $s \in \Types^d$ of types.
Define a \defn{relation $r$ of signature $s \in \Types^d$ with values in a space $\Space$} to be a function from $\Nats^d$ to $\Space$.
\end{definition}

We may encode a relation $r$ with signature $s \in \Types^d$ as an array $X^r \defas (X^r_{k})_{k \in \Nats^d}$ given by
\[
X^r_{k} = r(k_1,\dotsc,k_d), \qquad \text{for } k = (k_1,\dotsc, k_d) \in \Nats^d.
\]

\begin{example} 
Let $\Types = \{\textit{users,movies}\}$.
A relation $r$ of signature $(\textit{users},\textit{movies})$ taking values in $\{1,2,3,4,5\}$ might record movie ratings.
This could be stored in an array, with rows corresponding to some enumeration of \textit{users}, and columns corresponding to some enumeration of \textit{movies}. 
A relation $r'$ of signature $(\textit{users},\textit{users})$ taking values in $\{0,1\}$ might store the symmetric friendship relations in a social network.
\end{example}

\begin{definition}[database]
Define a \defn{database} to be a collection of $R$ relations $r^1,\dotsc,r^R$ of signature $s^1,\dotsc,s^R \in T^{d_1},\dotsc,T^{d_R}$, taking values in spaces $\Space^1,\dotsc,\Space^R$, respectively.
\end{definition}

We may encode a database as a collection of arrays $(X^{r^j})_{j=1}^R$, where $X^{r^j}$ encodes the relation $r^j$.  
We will often refer to the collection of arrays $(X^{r^j})_{j=1}^R$ as if it were the database itself.

Permuting the ordering of objects within a database can result in a permutation of the indices of several of the arrays encoding its relations.
For each type $t \in \Types$, let $p_t \in \SGinf$ be a permutation of $\Nats$. 
Write $p = (p_t ; t \in \Types) \in \SGinf^T$ for a collection of such permutations.
Given a signature $s \in \Types^{d}$, define $p^s$ to be the map from $\Nats^{d}$ to $\Nats^{d}$ such that
\[
p^s(k) \defas (p_{s_1}(k_1), \dotsc,p_{s_{d}}(k_{d})), \qquad \text{for } k \in \Nats^{d}.
\]
In other words, $p^s$ maps a sequence $k_1,\dotsc,k_d$ of indices (indexing objects of type $s_1,\dotsc,s_d$, respectively) to the sequence where each index is permuted by the permutation corresponding to its type.

If $X^r$ is the encoding of a relation $r$ with signature $s \in \Types^d$, then the permuted relation $r \circ p$ is represented by the array $X^{r\circ p}$ given by
\[
X^{r \circ p}_{k} = X^r_{p^s(k)}, \qquad \text{for } k \in \Nats^d. \label{eq:arrays:arrayperm}
\]

\begin{definition}[exchangeable database]
\label{def:arrays:exdatabase}
We say that a random database $(X^{r^j})_{j=1}^R$ is \emph{exchangeable} when it has the same distribution as $(X^{r^j\circ p})_{j=1}^R$ for every $p \in \SGinf^T$.
\end{definition}

%If the ordering of all objects is arbitrary, then $X^r$ is $\pi$-exchangeable where $\pi$ is the partition of consecutive integers with lengths $m^r_1, m^r_2, \ldots, m^r_O$.

%\PROBLEM{Here}

\subsection{A simplified representation theorem}

%Let $\Law(Y)$ be the law (distribution) of a random variable $Y$ and define $\chi_m X \defas (X_{i_1\ldots i_d}; \ i_j \le m )$.
The following result characterises the distribution of any exchangeable database to arbitrary accuracy.

\begin{cor}[simple functional representation for exchangeable databases]
  \label{cor:simple-database}
   Let $(X^{r^j})_{j=1}^R$ be an exchangeable random database.
   Then there exists a sequence of random measurable functions $F^{j,1}, F^{j,2}, \dotsc$ for 
   every $j=1,\ldots, R$ and collection of \iid Uniform$[0,1]$ random variables $(\AHvar^t_i)_{i\in\Nats,t \in \Types}$ such that 
   the random databases $(X^{r^j,n})_{j=1}^R$
    converge in distribution to $(X^{r^j})_{j=1\ldots R}$ on all finite subarrays as $n\to \infty$, where   
   the sequence of arrays $X^{r^j,1},X^{r^j,2},\dotsc$ for $j = 1,\dotsc,R$ are given by
   \[
     X^{r^j,n}_{k} := F^{j,n}(\AHvar^{s^j_1}_{k_1},\dotsc,\AHvar^{s^j_{d_j}}_{k_{d_j}}), \qquad \text{for } k \in \Nats^{d_j}.
   \]
\end{cor}

This is a corollary of theorem~\ref{thm:simple-database}, presented later in this chapter, which states that the distributions of finite subarrays are mutually absolutely continuous and the associated Radon--Nikodym derivatives converge uniformly to $1$ as $n \to \infty$.
We also present an almost-sure representational result which requires some heavy notation which we build up in subsequent sections.

To demonstrate corollary~\ref{cor:simple-database}, we present two special cases applicable to modelling a network with side information for each node and a social network with associated user-item data.

\begin{cor}
  \label{cor:network-side-simple}
  Consider an exchangeable database with one object type, one unary relationship, and one binary relationship; denote the binary relationship by the array $X=(X_{ij})_{i,j\in\Nats}$ and the unary relationship with the sequence $C=(C_i)_{i\in\Nats}$.
   Then there exists a sequence of pairs of random measurable functions $(F^n, G^n)_{n\in\Nats}$ and a collection of \iid Uniform$[0,1]$ random variables $(\AHvar_{i})_{i\in\Nats}$ such that if we define the arrays $X^1,X^2,\dotsc$ and sequences $C^1,C^2,\dotsc$ by
   \[ 
     X^n_{i,j} &\defas F^n(\AHvar_{i},\AHvar_{j}), \qquad \text{for } i,j,n\in\Nats, \\
     C^n_{i} &\defas G^n(\AHvar_{i}), \qquad \text{for } i,n\in\Nats,
    \]
   then $(X^n,C^n)$ converges in distribution to $(X,C)$ as $n \to \infty$ on all finite subarrays.
\end{cor}

\begin{cor}
  Consider an exchangeable database with two object types, one binary relation between two objects of the first type and a binary relation between objects of different types;  denote the first relation by the array $X=(X_{ij})_{i,j\in\Nats}$ and the second by $Y=(Y_{ik})_{i,k\in\Nats}$.
   Then there exists a sequence of pairs of random measurable functions $(F^n, G^n)_{n\in\Nats}$ and a collection of \iid Uniform$[0,1]$ random variables $(\AHvar_{i})_{i\in\Nats}, (\AHvaralt_{i})_{i\in\Nats}$ such that if we define the arrays $X^1,X^2,\dotsc$ and  $Y^1,Y^2,\dotsc$ by
   \[ 
     X^n_{ij} &\defas F^n(\AHvar_{i},\AHvar_{j}), \qquad \text{for } i,j,n\in\Nats, \\
     Y^n_{ik} &\defas G^n(\AHvar_{i}, \AHvaralt_{k}), \qquad \text{for } i,k,n\in\Nats,
    \]
   then $(X^n,Y^n)$ converges in distribution to $(X,Y)$ as $n \to \infty$ on all finite subarrays.
\end{cor}

%\begin{rem}[uniform distributions]\label{rem:uniform}
%The uniform distributions in the theorem are canonical but the theorem still holds with any non-atomic probability measure on a Borel space \eg Gaussian distributions.
%\end{rem}

\begin{rem}[random functions]\label{rem:randfunc}
  When we refer to a random function we mean a deterministic function which takes an additional source of randomness as input.
  For example $F(\textrm{arguments}) \defas f(U, \textrm{arguments})$ where $f$ is deterministic and $U$ is uniformly distributed would be an example of a random function.
  This is made precise in the statements of the main results.
\end{rem}

\subsection{Interpretation and examples}

Corollary~\ref{cor:simple-database} states that the joint distribution of an exchangeable database can be arbitrarily well approximated by a collection of random measurable functions and uniform random variables.
This functional form provides a set of parameters to be estimated that are naturally hierarchical.
The functions $(F^{j,n})$ capture properties of entire relations whilst the $(U^t_i)$ represent particular objects underlying the relational data.

\subsubsection{Example : Exchangeable networks}

Consider modelling a single binary relation which indicates whether or not two nodes in a network are connected or not.
This data is typically represented in the form of an adjacency matrix $(X_{ij})$ where $X_{ij} = 1$ if and only if node $i$ is connected to node $j$.
Theorem~\ref{thm:simple-database} states that if the distribution of $X$ is exchangeable then it can be arbitrarily well approximated by
\begin{equation}
(F(\AHvar_i, \AHvar_j))
\end{equation}
where $F$ is a random measurable function and $(U_i)$ are \iid Uniform$[0,1]$ random variables.%\footnotemark{}.
This special case was used previously by \cite{Hoff2007-ja,Roy2009-ge,Lloyd2012-sb} and in chapter~\ref{ch:networks} to inspire probabilistic models of networks of the form
\begin{eqnarray}
(\AHvar_i) & \probgetsiid & \textrm{\eg Gaussian} \\
F & \probgets & \textrm{\eg Gaussian process, bilinear function}\ldots \\
W_{ij} & \deterministicgets & F(\AHvar_i, \AHvar_j) \\
X_{ij} & \probgets & \textrm{Bernoulli}(\sigma(W_{ij}))\label{eq:graphon}.
\end{eqnarray}
This is illustrated in figure~\ref{fig:graphon}; in this case $F$ can be interpreted as a blurred adjacency matrix.

%\footnotetext{Note that we occasionally omit the index sets of collections of random variables for reduce notational clutter. For example $(U_i)$ is short hand for $(U_i)_{i\in\Nats}$.}

\begin{figure}[ht]
\centering
\begin{tabular}{c}
\input{\arraysfigsdir/fig_graphon_wide}
\end{tabular}
\caption[Illustration of the Aldous--Hoover representation of a network.]{
Illustration of a model for network data inspired by the Aldous--Hover representation theorem.
The left shows a random sample of a binary network (represented by an adjacency matrix) generated by a model of the form given by equation~\eqref{eq:graphon}.
}
\label{fig:graphon}
\end{figure}

\subsubsection{Example : A simple database}

Consider the database shown on the left hand side of figure~\ref{fig:multi-rel-seq}.
There are two objects, students and courses, and three relations, the unary relation `age' acting on students, the binary relation `friends' acting on pairs of students and the binary relation `grade' that acts on students and courses.
Sample data encoded in arrays is shown at the bottom of this figure.

Exchangeability of this database means that the entries of the leftmost table, the rows and columns of the second table and the rows of the third table can be arbitrarily permuted without changing the distribution of the database when viewed as a random variable.
Similarly the columns of the rightmost table may be independently arbitrarily permuted.

The functional form resulting from the application of theorem~\ref{thm:simple-database} or corollary~\ref{cor:simple-database} to this data structure is shown on the right hand side of figure~\ref{fig:multi-rel-seq}.
The two objects are represented by \iid random variables, $(U_i)$ for students, $(V_i)$ for courses and the three relations are represented by three random functions $F,G$ and $H$ whose inputs are the random variables representing the objects the relations act upon.

\begin{figure}[ht]
\centering
\begin{tabular}{cc}
\tiny \begin{tikzpicture}[scale=0.5, node distance = 4cm, auto]
  % Define block styles
  \tikzstyle{decision} = [diamond, draw, fill=blue!20, 
      text width=4.0em, text badly centered, node distance=1.0cm, inner sep=0pt]
  \tikzstyle{invisible} = [diamond, draw=white, fill=white, 
      text width=4.0em, text badly centered, node distance=1.0cm, inner sep=0pt]
  \tikzstyle{block} = [rectangle, draw, fill=blue!20, 
      text width=4.0em, text centered, rounded corners, minimum height=2em]
  \tikzstyle{placeholder} = [rectangle, draw, 
      text width=5em, text centered, rounded corners, minimum height=2em]
  \tikzstyle{line} = [draw, -latex']
  \tikzstyle{cloud} = [draw, ellipse,fill=red!20, node distance=0.75cm,
      minimum height=1em]
  \begin{scope}[yshift=0cm]
    % Place nodes
    \node [block] (student) at (-2.5,1) {Student};
    %\node [block, right of=student] (course) {Course};
    \node [block] (course) at (+2.5,1) {Course};
    \node [decision] at(+2.5, -1.5) (takes) {Takes};
    \node [cloud] (friends) at (-2.5, -4.0) {Friends};
    \node [cloud] (grade) at (+2.5, -4.0) {Grade};
    \node [cloud] (age) at (-5.5, -4.0) {Age};
    % Draw edges
    \path [line] (student) -- (takes);
    \path [line] (course) -- (takes);
    \path [line] (takes) -- (grade);
    %\path [line] (student) -- (observed);
    %\path [line] (observed) -- (friends);
    \draw[->] (student.south) .. controls (-1.0,-1) and (-1.0,-2) .. (friends.north east);
    \draw[->] (student.south) .. controls (-4.0,-1) and (-4.0,-2) .. (friends.north west);
    \path [line] (student.south west) -- (age.north east);
  \end{scope}
  \begin{scope}[yshift=-5cm]
    \draw (-4,0) --(-4,-3) --(-1,-3) --(-1,0) --(-4,0);
    \draw (0,0) --(0,-3) --(5,-3) --(5,0) --(0,0);
    \draw (-6.5,0) --(-6.5,-3) --(-4.5,-3) --(-4.5,0) --(-6.5,0);
    \draw (0,0) --(0,-3) --(5,-3) --(5,0) --(0,0);
    \node at (-3.25, -0.25) {\checkmark};
    \node at (-2.75, -0.25) {\checkmark};
    \node at (-2.25, -0.25) {\checkmark};
    \node at (-1.75, -0.25) {\checkmark};
    \node at (-1.25, -0.25) {\checkmark};
    \node at (-3.75, -0.75) {\checkmark};
    \node at (-3.75, -1.25) {\checkmark};
    \node at (-3.75, -1.75) {\checkmark};
    \node at (-3.75, -2.25) {\checkmark};
    \node at (-3.75, -2.75) {\checkmark};
    \node at (-2.75, -0.75) {$\times$};
    \node at (-3.25, -1.25) {$\times$};
    \node at (-1.25, -1.75) {\checkmark};
    \node at (-2.25, -2.75) {\checkmark};
    
    \node at (0.25, -0.25) {A};
    \node at (4.25, -0.75) {A};
    \node at (3.75, -1.25) {B};
    \node at (1.25, -2.75) {B};
    \node at (0.75, -1.25) {C};
    \node at (2.75, -2.75) {C};
    \node at (2.25, -1.25) {D};
    \node at (2.25, -1.75) {D};
    \node at (3.75, -0.75) {E};
    \node at (4.75, -1.25) {F};
    
    \node at (-5.5, -0.25) {15};
    \node at (-5.5, -0.75) {15};
    \node at (-5.5, -1.25) {15};
    \node at (-5.5, -1.75) {14};
    \node at (-5.5, -2.25) {14};
    \node at (-5.5, -2.75) {16};
  \end{scope}
\end{tikzpicture}
 & \tiny \input{\arraysfigsdir/multi_rel_seq_U_F}
\end{tabular}
\caption[Illustration of a representation of an exchangeable database.]{Left: A pictorial representation of a relational database. Right: The functional representation of the distribution of data of this form guaranteed to be an arbitrarily good approximation by theorem~\ref{thm:simple-database}}
\label{fig:multi-rel-seq}
\end{figure}

\section{Representation theorems for exchangeable databases}
\label{sec:proof_database}

We now state and prove the main technical results of this chapter.

\subsection{Background: $\pi$-exchangeability}

We first summarise material from \citet{Kallenberg1999-pj}.
Let $\pi$ be a partition of the set $\{1,\dots,d\}$ and let $\pi_i \defas (I \in \pi : i \in I$) \ie this is the partition element which $i$ is in.
Let $p = (p^{\pi_i} : i = 1,\dots,d)$ be a collection of permutations of $\Nats$ \ie a list of permutations which are equal on the partition elements.
We say that an array $X$ is $\pi$-exchangeable if 
\begin{equation}
  %X^{r\circ p} \eqd X
  (X_k) \eqd (X_{p(k)})
\end{equation}
for all collections $p$ of permutations of $\Nats^d$ of the form given above.
This matches the symmetry of an exchangeable database defined in definition~\ref{def:arrays:exdatabase} and equation~\eqref{eq:arrays:arrayperm} if $X$ represents a relation with signature $(\pi_i : i = 1,\dots,d)$.
%This is precisely the type of symmetry we defined for exchangeable databases 

We say that a $\pi$-exchangeable array $X$ is \emph{simple} if it admits a functional representation of the form
\begin{equation}
  X_{k} = f(U, U^{\pi_1}_{k_1}, \dots, U^{\pi_d}_{k_d})
\end{equation}
where $f$ is a measurable function and $U, (U^{\pi_j}_{k_j})$ are \iid Uniform$[0,1]$ random variables.

\begin{prop}[$\pi$-exchangeability representation theorem]
  \label{thm:piex}
  Let $X$ be a $\pi$-exchangeable array.
  Then there exist some simple $\pi$-exchangeable arrays $X_1$ , $X_2$ , \dots such that
  $\chi_m X_n$ and $\chi_m X$ are mutually absolutely continuous for all $m,n \in \Nats$ and the associated Radon--Nikodym derivatives tend
  uniformly to 1 as $n \to \infty$ for fixed $m$ where $\chi_m$ is the array subset operation such that $\chi_m Y \defas (Y_{k} : k \in \{1,\dots,m\}^d)$.
\end{prop}

\subsection{Simple exchangeable database representation theorem}

We generalise the concept of a simple $\pi$-exchangeable array by saying that an exchangeable database $(X^{r_j})_{j=1}^R$ is simple if it admits a functional representation of the form
\begin{equation}
  X^{r^j}_{k} = f^{r^j}(U, U^{s^{j}_1}_{k_1},\dots,U^{s^{j}_{d_j}}_{k_{d_j}}) \,\, \forall \,\, k \in \Nats^{d_j} \,\, \forall \,\, j \in {1,\dots,R}.
\end{equation}

We first state a simple but useful lemma.
It can be proven by writing down the definitions of Radon--Nikodym derivatives and uniform convergence.

\begin{lem}
  \label{lem:contractionrnd}
  Suppose that two sequences of random variables $(X^n)$ and $(Y^n)$ are such that $X^n$ and $Y^n$ are mutually absolutely continuous for all $n$, and the associated Radon--Nikodym derivatives converge uniformly to 1.
  Then for any measurable $f$, $f(X^n)$ and $f(Y^n)$ are mutually absolutely continuous for all $n$ and the associated Radon--Nikodym derivatives converge uniformly to 1.
\end{lem}

We also build up some notation for a particular exchangeable database $(X^{r^j})_{j=1}^R$.%\fTBD{Any way to make this notation easier?}.
Assume w.l.o.g.\ that the types are ordered (\eg type 1, type 2,\dots) and the types in each signature are sorted according to this ordering (\eg $s = (1,1,2,3,3,\dots)$).
Define the multiplicity of a type $t$ in signature $s$ by the number of times the type appears in the signature, and denote this by $m_{t}^s$.
Define the maximum multiplicity of a type $t$ in a database as $m_t = \max_j m_t^{s^j}$.
Let $d = \sum_{t\in T} m_t$ be the sum of the maximum multiplicities.
Let $\pi$ be the partition of $\{1,\ldots,d\}$ consisting of consecutive blocks of integers with sizes equal to $\{m_1,\ldots,m_T\}$.
Let $\rho_j : \Nats^d \to \Nats^{d_j}$ be the projection that keeps the first $m_{1}^{s^j}$ dimensions, ignores the next $m_1 - m_{1}^{s^j}$, keeps the next $m_{2}^{s^j}$ et cetera.
Similarly, let $\tilde{\rho}_j : \Nats^d \to \Nats^d$ be analogous to $\rho_j$ except that the indices removed by $\rho_j$ are now set to the value 1.
Again similarly, for a signature $s$ let $I_s$ be the subset of $\{1,\ldots,d\}$ that contains the first $m_1^s$ integers and then not the next $m_1 - m_1^s$ integers, then the next $m_2^s$ integers, and then not the next $m_2 - m_2^s$ integers and so on.
Let $\mathbb{I}_s$ be the indicator function of the set $I_s$ within $\{1,\ldots,d\}$.
We can now state and prove the first main result of this chapter.

\begin{thm}[Simple exchangeable database representation theorem]
  \label{thm:simple-database}
  Let $(X^{r^j})_{j=1}^R$ be an exchangeable database.
  Then there exist some simple exchangeable databases $(X^{r^j,1})_{j=1}^R$ , $(X^{r^j,2})_{j=1}^R$ , \dots such that
  $(\chi_m X^{r^j,n})_{j=1}^R$ and $(\chi_m X^{r^j})_{j=1}^R$ are mutually absolutely continuous for all $m,n \in \Nats$ and the associated Radon--Nikodym derivatives converge uniformly to 1 as $n \to \infty$ for fixed $m$.
\end{thm}

\begin{proof}
  We first embed all of the arrays representing the exchangeable database into one larger array:
  \[
    R_{k} = (X^{r^j}_{\rho_j(k)} : j = 1,\dots,R) \quad \forall \, k \in \Nats^d.
  \]
  By the definition of an exchangeable database, this array is $\pi$ exchangeable where $\pi$ is as defined in the preceding text.
  By proposition~\ref{thm:piex} there exists a sequence of measurable functions $(f^n)$ and collection of independent uniform random variables $U$ and $(U^t_i)$ such that $(\chi_m R_{k})$ and $(\chi_m f^n(U, U^{\pi_1}_{k_1}, \dots, U^{\pi_d}_{k_d}))$ are mutually absolutely continuous and the associated Radon--Nikodym derivates converge uniformly to 1 as $n \to \infty$ for all $m$.
  By using a projection in lemma~\ref{lem:contractionrnd} we have the same convergence for the following subarrays (jointly over $j$)
  \[
    f^{n,j}(U, U^{\pi_1}_{\tilde{\rho}_j(k_1)}, \dots, U^{\pi_d}_{\tilde{\rho}_j(k_d)}) \to X^{r^j}_{\rho_j(k)}
  \]
  and by exchangeability we also have \[
    f^{n,j}(U, U^{\pi_1}_{\tilde{\rho}_j(k_1+1)}, \dots, U^{\pi_d}_{\tilde{\rho}_j(k_d+1)}) \to X^{r^j}_{\rho_j(k)}.
  \]
  By the Borel isomorphism theorem there is a bimeasurable function $h$ and a uniformly distributed random variable $V$ such that
  \[
    V = h(U, U^1_1, \dots, U^T_1) \,\, \as
  \]
  Let $V^t_k = U^t_{k+1}$ and
  \[
    g^{n,j}(V, V^{\pi_i}_{k_i} : i \in I_{s^j}) \defas f^{n,j}(U, U^{\pi_1}_{\tilde{\rho}_j(k_1+1)}, \dots, U^{\pi_d}_{\tilde{\rho}_j(k_d+1)}).
  \]
  $g$ is measurable by composition of measurable $f$ and $h$.
  Finally set
  \[
    X^{r_j,n}_{k} \defas g^{n,j}(V, V^{\pi_i}_{k_i} : i \in I_{s^j})
  \]
  and we are done.
\end{proof}

\subsection{Almost sure exchangeable database representation theorem}

To prove an almost sure representation theorem for exchangeable databases we will require yet more notation; we follow Kallenberg except where we think changes may make it easier to understand.
There is only so much we have been able to do to reduce the notational burden so we provide corollaries of the theorem after its statement and proof to demonstrate that behind the notation are some fairly intuitive concepts.

We define for each $k \in \Nats^d$ and $I \subset \{1,\ldots,d\}$ a set $k_I \subset \Nats$ by
\begin{equation}
k_I = \{k_i: i \in I\}, \quad k \in \Nats^d
\end{equation}
\ie $k_I$ only includes the indices in $I$ and is a set.

We further define $k_{\pi I}$ by
\begin{equation}
  k_{\pi I} = (k_{I \cap J} : J \in \pi)
\end{equation}
N.B. we have replaced Kallenberg's function notation with a list since this may be more natural to the readers of this work.
Let $\Nats^{\leq d}$ be the collection of all subsets of the naturals $K \subset \Nats$ with cardinality $|K| \leq d$.

Recall that for a signature $s$ we define $I_s$ to be the subset of $\{1,\ldots,d\}$ that contains the first $m_1^s$ integers and then not the next $m_1 - m_1^s$ integers, then the next $m_2^s$ integers, and then not the next $m_2 - m_2^s$ integers and so on.
%Let $k^s \defas (k_i : i \in I_s)$ and let $\pi_s \defas \{J \cap I_s : J \in \pi\}$
%Let $\pi_s \defas \{J \cap I_s : J \in \pi\}$

We refer to indexed collections of \iid Uniform$[0,1]$ random variables as $U$-arrays.

\begin{prop}[\citet{Kallenberg1999-pj}]
\label{prop:piexas}
  Let $X$ be a random $d$-array in a Borel space $S$.
  Then $X$ is $\pi$-exchangeable iff there exists a measurable function $f:[0,1]^{2^d}\to S$ and a $U$-array $\xi$ with index set $\prod_{J\in\pi} \Nats^{\leq |J|}$ such that
  \begin{equation}
    X_k = f(\xi_{k_{\pi I}} : I \subset \{1,\ldots,d\}) \ \as, \quad k \in \Nats^d.
    \label{eq:piexas}
  \end{equation}
\end{prop}

We now extend this proposition to the case of an exchangeable database.
Again we state a useful lemma.

\begin{lemma}
  \label{lemma:copies-full-simple}
  Let $X_{ij} = f({U}_i, {V}_{ij}) \ \as \ i,j \in \Nats$ where $f$ is measurable and ${V}_{ij}$ are $\iid$
  Suppose further that $X_{ij} = X_i \ \as$ for some collection of random variables $X_{i} \, \forall \, i,j \in \Nats$.
  Then there exists a measurable function $g$ such that $X_i = g({U}_i) \ \as$
\end{lemma}

\begin{proof}
  For each $i$ and conditioned on $U_i$, ${X}_{ij}$ are \iid
  Then by the conditional strong law of large numbers
  \[
    \frac{1}{n}\sum_{j \leq n} f(U_i, V_{ij}) \to \mathbb{E}(f(U_i, V_{ij})\given U_i) \,\, \as
  \]
  whence
  \[
    \frac{1}{n}\sum_{j \leq n} f(U_i, V_{ij}) = X_i = \mathbb{E}(f(U_i, V_{ij})\given U_i) =: g(U_i) \,\, \as
  \]
\end{proof}

\begin{thm}[almost sure functional representation for exchangeable databases]
  \label{thm:as-database}
  A database $(X^{r^j})_{j=1}^R$ is exchangeable iff there exist measurable functions $f^j : [0,1]^{2^{d_j}} \to S^j$ and a $U$-array $\xi$ with index set $\prod_{J\in\pi} \Nats^{\leq |J|}$ such that
  \begin{equation}
    X_{\rho_j(k)}^{r^j} = f^j(\xi_{k_{\pi I}} : I \subset I_{s^j}) \ \as, \quad k \in \Nats^d, \ j \in \{1,\ldots,R\}.
  \end{equation}
\end{thm}

\begin{proof}
  It is simple to verify that this construction produces databases with the required symmetry properties so we need only show that such a construction is guaranteed to exist by exchangeability.
  We proceed as before by embedding all of the arrays into one larger array
  \[
    R_{k} = (X^{r^j}_{\rho_j(k)} : j = 1,\dots,R) \quad \forall \, k \in \Nats^d
  \]
  which is $\pi$ exchangeable and thus by proposition~\ref{prop:piexas} there exists a measurable $g$ and $U$-array $\xi$ with index set $\prod_{J\in\pi} \Nats^{\leq |J|}$ such that
  \[
    R_{k} = g(\xi_{k_{\pi I}} : I \subset \{1,\ldots,d\}) \ \as, \quad k \in \Nats^d.
  \]
  We then repeatedly apply lemma~\ref{lemma:copies-full-simple} to remove the redundancy of this representation.
  For example, for some $i \notin I_{s^j}$ we have that there exists some measurable $h^j$ such that
  \[
    h^j(\xi_{k_{\pi I}} : I \subset \{1,\ldots,d\} \backslash i) = X^{r_j}_{\rho_j(k)} \ \as
  \]
  After repeated application we will find that there exists measureable $f^j$ such that
  \[
    f^j(\xi_{k_{\pi I}} : I \subset I_{s^j}) = X^{r_j}_{\rho_j(k)} \ \as
  \]
  and we are done.
\end{proof}

\subsection{Examples}

\begin{cor}
  Consider an exchangeable database with one object type, one unary relationship, and one binary relationship; denote the binary relationship by the array $X=(X_{ij})_{i,j\in\Nats}$ and the unary relationship with the sequence $C=(C_i)_{i\in\Nats}$.
   Then there exist measurable functions $(f, g)$ and a collection of \iid Uniform$[0,1]$ random variables $U, (\AHvar_{i})_{i\in\Nats}, (\AHvar_{\{ij\}})_{i,j\in\Nats}$ such that
   \[ 
     X_{ij} & = f(U, \AHvar_{i},\AHvar_{j}, \AHvar_{\{ij\}}), \qquad \text{for } i,j\in\Nats, \\
     C_i & = g(U, \AHvar_{i}), \qquad \text{for } i\in\Nats,
    \]
almost surely.
\end{cor}

This is the Aldous--Hoover representation for a jointly exchangeable array and the de Finetti representation of an exchangeable sequence at the same time, with coupled uniform random variables.

\begin{cor}
  Consider an exchangeable database with two object types, one binary relation between two objects of the first type and a binary relation between objects of different types;  denote the first relation by the array $X=(X_{ij})_{i,j\in\Nats}$ and the second by $Y=(Y_{ik})_{i,k\in\Nats}$.
   Then there exists a pair of measurable functions $(f, g)$ and a collection of \iid Uniform$[0,1]$ random variables $U, (\AHvar_{i})_{i\in\Nats}, (\AHvaralt_{i})_{i\in\Nats}, (\AHvar_{\{ij\}})_{i,j\in\Nats}, (W_{ij})_{i,j\in\Nats}$ such that
   \[ 
     X_{ij} & = f(U,\AHvar_{i},\AHvar_{j},  \AHvar_{\{ij\}}),  \qquad &\text{for } i,j\in\Nats, \\
     Y_{ik} & = g(U, \AHvar_{i}, \AHvaralt_{k}, W_{ik}), \qquad &\text{for } i,k\in\Nats,
    \]
almost surely.
\end{cor}

This is the Aldous--Hoover representation for a jointly exchangeable array and the Aldous--Hoover representation of a separately exchangeable array at the same time, but with coupled uniform random variables due to the shared objects underlying the data.

\begin{cor}
  Consider an exchangeable database with two object types, one binary relation between two objects of the first type and a ternary relation between two objects of the first type and one of the second type;  denote the first relation by the array $X=(X_{ij})_{i,j\in\Nats}$ and the second by $Y=(Y_{ijk})_{i,j,k\in\Nats}$.
   Then there exists a pair of measurable functions $(F, G)$ and a collection of \iid Uniform$[0,1]$ random variables $U, (\AHvar_{i})_{i\in\Nats}, (\AHvaralt_{i})_{i\in\Nats}, (\AHvar_{\{ij\}})_{i,j\in\Nats}, (W_{ij})_{i,j\in\Nats},  (Z_{\{ij\}k})_{i,j,k\in\Nats}$ such that
   \[ 
     X_{ij} & = F(U,\AHvar_{i},\AHvar_{j},  \AHvar_{\{ij\}}),  \qquad &\text{for } i,j\in\Nats, \\
     Y_{ijk} & = G(U, \AHvar_{i}, U_j, \AHvaralt_{k}, \AHvar_{\{ij\}}, W_{ik}, W_{jk}, Z_{\{ij\}k}), \qquad &\text{for } i,j,k\in\Nats,
    \]
almost surely.
\end{cor}

This is the Aldous--Hoover representation for a jointly exchangeable array and the Kallenberg representation of a $\pi$-exchangeable array, where $\pi = \{\{1,2\},3\}$, at the same time, but again with shared uniform random variables.

\section{A generic statistical model template}

In analogy to the work of \cite{Hoff2007-ja, Roy2009-ge, Lloyd2012-sb} on exchangeable arrays, theorem~\ref{thm:simple-database} naturally inspires a generic generative model of exchangeable databases.
Each object of type $t$ in the database is associated with an \iid sample, $\AHvar_i^t$, from some distribution $\mathcal{U}$ \eg Uniform, Gaussian.
For each relation $r^j$ we sample a random function $F^j$ from some distribution $\mathcal{F}^j$, \eg Gaussian process, random (bi/tri/\ldots)linear functions.
We denote the evaluation of these functions at the corresponding values of $(\AHvar_i^t)$ by $W^j$.
$W^j$ can then be passed through a link function and distribution $L^j(\cdot)$ to model the observed value of the relation $r^j$.
\[
(\AHvar_i^t) & \probgetsiid  \mathcal{U} \\
F^j & \probgets  \mathcal{F}^j \\
W^j_{k} & \deterministicgets F^j(\AHvar^{s^j_1}_{k_1},\cdots,\AHvar^{s^j_{d_j}}_{k_{d_j}}) \\
X^{r_j}_{k} & \probgets  L^j(W^j_{k}) \qquad \text{independently across $j$ and $k$.}
\]

\begin{rem}[dependence between functions]
In general, the functions $F^j$ may be dependent.
However, the representation results presented in this chapter provide no guidance on the form of this dependence; stronger assumptions than exchangeability would have to be made.
In practice one may model the functions to be independent a priori at the risk of wasting statistical strength.
\end{rem}

\subsection{Deriving appropriate forms for conditional models}

Consider a social network with side information on each user.
The natural generative model for such data inspired by the simple database representation result is
\[
  U_i & \probgetsiid \mathcal{U} \\
  F & \probgets \mathcal{F} \\
  G & \probgets \mathcal{G} \\
  X_{ij} & \probgets L^X(F(U_i, U_j)) \\
  C_{i} & \probgets L^C(G(U_i))
\]
but in the situation where $C$ is fully observed and the task is to predict missing entries of $X$ it may be unnecessary to learn a model of $C$.
Indeed, if $C$ is assumed to be noiselessly observed, then we have $C_i = L^C(G(U_i))$ where $L^C(G(.))$ is a deterministic function.
We can then choose to model $X$ as follows
\[
  F'(U_i, U_j, C_i, C_j) & = F'(U_i, U_j, L^C(G(U_i)), L^C(G(U_j))) = F(U_i, U_j) \\
  X_{ij} & \probgets L^X(F'(U_i, U_j, C_i, C_j)).
\]
Furthermore we could choose $F'$ to not depend on $U_i, U_j$ resulting in the model
\[
  X_{ij} & \probgets L^X(F''(C_i, C_j))
\]
which is simply a regression model.

Indeed, the same logic can be applied to exchangeable sequences to derive more standard regression models.
Suppose that $(X_i, Y_i)$ is an exchangeable sequence.
By de Finetti's theorem we can model this as
\[
  U_i & \probgets \mathcal{U} \\
  F & \probgets \mathcal{F} \\
  G & \probgets \mathcal{G} \\
  X_i & \probgets L^X(F(U_i)) \\
  Y_i & \probgets L^Y(G(U_i))
\]
and by similar arguments to those above, if we were to assume that $X$ was noiselessly observed we could model $Y$ as
\[
  Y_i & \probgets L^Y(G(X_i))
\]
which is a standard regression model, or
\[
  Y_i & \probgets L^Y(G(U_i, X_i))
\]
which combines latent variable modelling and regression.
An example of this hybrid model structure can be found in \cite{Wang2012-rc} but is surprisingly uncommon.

Now suppose we perfectly observe a binary relation; what can this tell us about modelling a related unary relation?
We now demonstrate how we could (almost) reconstruct an example model presented in \cite{Friedman1999-mo}.
They present a simple genetic model where one's maternal chromosome depends on one's mother's maternal and paternal chromosomes and similarly for one's paternal chromosome.
In this example we have two binary relations, mother-of and father-of and two unary relations, maternal and paternal chromosomes.
We write this as $X_{ij} = 1 \iff$ person $i$ is the mother of person $j$ and $Y_{ij} = 1 \iff$ person $i$ is the father of person $j$ and the array elements take the value 0 otherwise.
Further let $M_i$ and $P_i$ be the maternal and paternal chromosomes of person $i$ respectively.
A potential form of a model for this data is
\[
  U_i & \probgets \mathcal{U} \\
  F & \probgets \mathcal{F} \\
  G & \probgets \mathcal{G} \\
  H & \probgets \mathcal{H} \\
  J & \probgets \mathcal{J} \\
  X_{ij} & \probgets L^X(F(U_i, U_j)) \\
  Y_{ij} & \probgets L^Y(G(U_i, U_j)) \\
  M_i & \probgets L^M(H(U_i)) \\
  P_i & \probgets L^P(J(U_i))
\]
Applying similar arguments to before we can replace $U_i$ by $U_i'  \defas (U_i, V_i, M_i, P_i)$ where we have split the latent variable $U_i$ into $U_i$ and $V_i$ for convenience.
We could then define
\begin{align}
L^X(F(U_i',U_j')) &= 
  \begin{cases}
    1 & \textrm{if } (M_i, P_i) = U_j \\
    0 & \textrm{otherwise}  
  \end{cases}\\
L^Y(G(U_i',U_j')) &=
  \begin{cases}
    1 & \textrm{if } (M_i, P_i) = V_j \\
    0 & \textrm{otherwise}  
  \end{cases}
\end{align}
Loosely speaking, if we performed inference in this model we would then find that in the posterior $U_i = (M_j, P_j)$ where person $j$ is the mother of person $i$ and similarly for $V_i$.
This would then allow the functions $H$ and $J$ to specify that the maternal chromosome of person $i$ depends probabilistically on the chromosomes of their mother and similarly for their paternal chromosome, recovering the structure of the probabilistic relational model.

However, the alert reader will have noticed that the functions $L^X(F(.,.))$ and $L^Y(G(.,.))$ defined above are almost everywhere zero.
Indeed the event $U_i = (M_j, P_j)$ could have measure zero and therefore cannot be said to be inferred without properly defining a suitable limiting process of valid inferences (see \eg \citet{Jaynes2003-jh} for many good examples where casual handling of measure zero events can lead to `paradoxes').
The functions $L^X(F(.,.))$ and $L^Y(G(.,.))$ are almost everywhere zero exactly because the relations mother-of and father-of define sparse graphs with in-degrees of at most one.
As remarked in chapter~\ref{ch:networks}, any not almost everywhere zero representing function will imply density when modelling a graph.
%The above exercise is certainly quite a contortion that is only clear when one knows the model one is aiming at.
We repeat that the construction of general representation results for sparse graphs is an area of emerging research and refer the reader to the sparse literature \citep{Lovasz2012-df, Wolfe2013-vs, Caron2014-on}.
% This example should reinforce the importance of finding good representation results for sparse 

As before with the regression example, it may be wise to use models which use both observed and hidden variables within the representing functions.
Examples are again rare but this has been applied in \eg the movie recommendation context \citep[e.g.][]{Menon2011-ku}.

\subsection{Higher order dependencies}

So far we have only considered probabilistic models inspired by the simple array versions of exchangeability theorems.
Do the almost sure representation theorems, with their more detailed latent variable representations, tell us anything useful?
Consider the following dataset: we have a social network $X_{ij}$ and a ternary relation $Y_{ijk}$ that records if both person $i$ and person $j$ have been to location $k$.
The simple array inspired representation of this data is of the form
\[
  X_{ij} & \probgets L^X(F(U_i, U_j)) \\
  Y_{ijk} & \probgets L^Y(G(U_i, U_j, V_k))
\]
and in contrast the almost sure result would suggest a more elaborate model of the form
\[
  X_{ij} & \probgets L^X(F(U_i, U_j, U_{\{ij\}})) \\
  Y_{ijk} & \probgets L^Y(G(U_i, U_j, U_{\{ij\}}, V_k, W_{ik}, W_{jk}, Z_{\{ij\}k})).
\]
Now suppose that $X$ included information on the type of relations in the social network, and that people $i$ and $j$ were in a romantic relationship.
This knowledge would make it more likely for this couple to visit \eg restaurants typically frequented by couples.
In the almost sure representation this information could easily be transferred from $X$ to the modelling of $Y$ via the shared latent variable $U_{\{ij\}}$.
In the simple array representation this knowledge would somehow have to be encoded in the latent variables $U_i$ and $U_j$ meaning that the problem of sharing this information is at least as hard as producing an effective generative model of which pairs of people are in a relationship\footnote{Presumably a problem that has been well studied by online dating services.}.

We are however unaware of any standard probabilistic models making use of this type of dependence.
This is most probably due to the potentially large computational barrier to storing and inferring the values of latent variables for each pair of objects.
There may however be use in this type of model on small datasets where one can afford relatively high computational costs to perform a more detailed analysis.

\subsection{A brief word about longitudinal data}

It is worth mentioning how the modelling paradigms presented here can be extended to longitudinal data (\ie time varying).
For example, consider longitudinal measurements of a social network $X_{ij}^t$.
Let $Y_{ij} \defas (X_{ij}^t : t \in \mathcal{T})$ where $\mathcal{T}$ represents some period of time.
Then $Y_{ij}$ can be assumed to be exchangeable meaning that we can represent its distribution as
\[
  Y_{ij} = G(U_i, U_j)
\]
which in turn means
\[
  X_{ij}^t = G^t(U_i, U_j)
\]
or more conventionally
\[
  X_{ij}^t = F(U_i, U_j, t)
\]
meaning that we can represent the distribution of the time evolving social network with fixed latent variables for each node but a time varying representing function.
The form of the time varying function is completely general, but it could be constrained further by assumptions of continuity or Markov exchangeability for example.
It is more typical however to assume that the latent variables also evolve through time (this still produces an exchangeable distribution) with the representing function potentially static.
Examples of models of this form include \citep[e.g.][]{Adams2010-ln, durante2014bayesian}.
It may be advantageous however to have fixed latent variables when one is jointly modelling both time evolving and static data.

\subsection{Prior work using models of this form}

In section~\ref{sec:networks:related} it was demonstrated that many models of single 2-arrays fit the form of the generic model presented above.
In particular there are models that assume $F$ is linear \citep[e.g.][]{Hoff2007-ja, Meeds2007-gd, Salakhutdinov2008-zt, Yu2008-tz, Miller2009-wg}, that $F$ is Gaussian process distributed \citep[e.g.][]{Lawrence2009-za, Yan2011-lc, Lloyd2012-sb} and other non-linear forms for $F$ both parametric \citep[e.g.][]{Hoff2002-vy} and nonparametric \citep[e.g.][]{Roy2009-ge}.
In addition to this there has been a line of work that uses increasingly more expressive forms of the distribution $\mathcal{U}$ \citep[e.g.][]{Wang1987-jd, Hoffman_undated-ri, Nowicki2001-xm, Kemp2006-jt, Xu2006-uy, Meeds2007-gd, Miller2009-wg, Palla2012-ch}.

Many, but not all, of these models have been extended to model $d$-arrays.
A summary of models using linear forms of $F$ is given in \cite{Kolda2009-ba}; non-linear models include \cite{Xu2012-ub}.

For full databases, the literature is limited to clustering / block / latent class models \citep{Kemp2006-jt, Xu2006-uy} and models using linear forms for the $F^r$ \citep[e.g.][]{Lippert2008-gg, Singh2008-cb, Singh2008-qw, Jimeng2009-rw, Acar2011-vg, Gao2011-ac, Nickel2011-pi, Acar2012-no, Ermis2012-gk, Shangguan2012-ga, Singh2012-jj, Acar2013-na, Andersen2013-rg, Yin2013-we}.

We should certainly expect to see new research using some of the more advanced forms proposed for networks being applied to higher order arrays or full databases.
However, rather than proposing yet more specific models it would be very interesting to see automated model building techniques applied to these domains, potentially combined with model forms inspired by probabilistic relational models.

\section{Discussion}

We have demonstrated how the concept of exchangeability can be applied to databases and used to derive a natural parameter space for statistical models of such data.
Identifying a parameter space is the first step in many statistical analyses, allowing either frequentist estimation of the parameters or Bayesian prior specification.
This concept is well established for exchangeable sequences where de Finetti's theorem applies.
For exchangeable arrays, the relevant representation theorems were presented by \citet{Aldous1981-lg} and \citet{Hoover1979-br} over 30 years ago but it is only recently that these results are being used to inspire probabilistic models \citep{Hoff2007-ja, Roy2009-ge, Lloyd2012-sb} and estimation procedures with frequentist guarantees \citep{Kallenberg1999-pj, Choi2013-th, Wolfe2013-vs}.
We hope that this work will continue and be extended to the analysis of exchangeable databases.

%\subsection{Mention sparsity again}

%\TBD{This is really important to talk about at all times! Reference latest literature on this topic.}

\outbpdocument{
\bibliographystyle{plainnat}
\bibliography{references.bib}
}
