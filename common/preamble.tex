% Paths to figure and table directories.
\newcommand{\introfigsdir}{figures/intro}
\newcommand{\constructionfigsdir}{figures/construction}
\newcommand{\descriptionfigsdir}{figures/description}
\newcommand{\gefcomfigsdir}{figures/gefcom}
\newcommand{\criticismfigsdir}{figures/criticism}
\newcommand{\networksfigsdir}{figures/networks}
\newcommand{\arraysfigsdir}{figures/arrays}

% Packages

\usepackage{booktabs}
\usepackage{tabularx}
\usepackage{multirow}
\usepackage{datetime}
\renewcommand{\tabularxcolumn}[1]{>{\arraybackslash}m{#1}}
\usepackage{relsize}
\usepackage{graphicx}
\usepackage{amsmath,amssymb,textcomp}
\usepackage{amsthm, bm}
\usepackage{nicefrac}
\usepackage{amsthm}
\usepackage{tikz}
\usetikzlibrary{shapes.geometric,arrows,chains,matrix,positioning,scopes,calc}
\tikzstyle{mybox} = [draw=white, rectangle]
\usepackage{common/picins}
\usepackage{etex}
\definecolor{camlightblue}{rgb}{0.601, 0.8, 1}
\usepackage{nth}
\usepackage{rotating}
\usepackage{array}
\usepackage{fp}

\usepackage[hyperpageref]{backref}
% Setup to show (pages 4 and 9) sort of thing in the bibliography - DD
%\def\foo{\hspace{\fill}\mbox{}\linebreak[0]\hspace*{\fill}}
%\def\foo{\parbox{3cm}{\hfill}
%\def\foo{\parbox{3cm}{\hfill}
%\newcommand\foo[1]{{\raggedleft{\hfill{\mbox{\hfill{#1}}}}}}
\newcommand{\comfyfill}[1]{% = Thorsten Donig's \signed
  \unskip\hspace*{0.1em plus 1fill}
  \nolinebreak[3]%
  \hspace*{\fill}\mbox{#1}
  \parfillskip0pt\par
}
\newcommand\foo[1]{{\comfyfill{\mbox{#1}}}}
%\newcommand\foo[1]{{\mbox{#1}}}
\renewcommand*{\backref}[1]{}
\renewcommand*{\backrefalt}[4]{%
\ifcase #1 %
%
\or
\foo{(page #2)}%
\else
\foo{(pages #2)}%
\fi
}

\usepackage{stringstrings}

\setlength{\marginparwidth}{0.8in}
%%%%%%%%%%%%%%%%%%%%%%%%%%%%%%%%%%%%%%%%%%%%%%%%%%%%%%%%%%
%%%% EDITING HELPER FUNCTIONS  %%%%%%%%%%%%%%%%%%%%%%%%%%%
%%%%%%%%%%%%%%%%%%%%%%%%%%%%%%%%%%%%%%%%%%%%%%%%%%%%%%%%%%

%% NA: needs attention (rough writing whose correctness needs to be verified)
%% TBD: instructions for how to fix a gap ("Describe the propagation by ...")
%% PROBLEM: bug or missing crucial bit 

%% use \fXXX versions of these macros to put additional explanation into a footnote.  
%% The idea is that we don't want to interrupt the flow of the paper or make it 
%% impossible to read because there are a bunch of comments.

%% NA's (and TBDs, those less crucially) should be written so 
%% that they flow with the text.

\definecolor{WowColor}{rgb}{.75,0,.75}
\definecolor{SubtleColor}{rgb}{0,0,.50}

% inline
\newcommand{\NA}[1]{\textcolor{SubtleColor}{ {\tiny \bf ($\star$)} #1}}
\newcommand{\LATER}[1]{\textcolor{SubtleColor}{ {\tiny \bf ($\dagger$)} #1}}
\newcommand{\TBD}[1]{\textcolor{SubtleColor}{ {\tiny \bf (!)} #1}}
\newcommand{\PROBLEM}[1]{\textcolor{WowColor}{ {\bf (!!)} {\bf #1}}}

% as margin notes

\newcounter{margincounter}
\newcommand{\displaycounter}{{\arabic{margincounter}}}
\newcommand{\incdisplaycounter}{{\stepcounter{margincounter}\arabic{margincounter}}}

\newcommand{\fTBD}[1]{\textcolor{SubtleColor}{$\,^{(\incdisplaycounter)}$}\marginpar{\tiny\textcolor{SubtleColor}{ {\tiny $(\displaycounter)$} #1}}}

\newcommand{\fPROBLEM}[1]{\textcolor{WowColor}{$\,^{((\incdisplaycounter))}$}\marginpar{\tiny\textcolor{WowColor}{ {\bf $\mathbf{((\displaycounter))}$} {\bf #1}}}}

\newcommand{\fLATER}[1]{\textcolor{SubtleColor}{$\,^{(\incdisplaycounter\dagger)}$}\marginpar{\tiny\textcolor{SubtleColor}{ {\tiny $(\displaycounter\dagger)$} #1}}}

%% For submission, make all render blank.
%\renewcommand{\LATER}[1]{}
%\renewcommand{\fLATER}[1]{}
%\renewcommand{\TBD}[1]{}
%\renewcommand{\fTBD}[1]{}
%\renewcommand{\PROBLEM}[1]{}
%\renewcommand{\fPROBLEM}[1]{}
%\renewcommand{\NA}[1]{}

% Not math

\def\ie{i.e.\ }
\def\eg{e.g.\ }
\def\etc{etc.\ }
\def\iid{i.i.d.\ }

\usepackage{nth}

% HUMBLE WORDS: shown slightly smaller when in normal text
% Thanks to Christian Steinruecken!

% HUMBLE WORDS: shown slightly smaller when in normal text
% Christian Steinruecken
%
\makeatletter%
%\def\@humbleformat#1{{\fontsize{}{1em}\selectfont #1}}
%\def\@humbleformat#1{\textsmaller{#1}}%
\newlength{\nonHumbleHeight}
\def\@humbleformat#1{{\settoheight{\nonHumbleHeight}{#1}\resizebox{!}{0.94\nonHumbleHeight}{#1}}}%
\def\@idxhumbleformat#1{{\relscale{0.95}{#1}}}%
%\def\@humbleformat#1{{#1}}%
\def\declareHumble#1#2{%
  \expandafter\def\csname #1\endcsname{\@humbleformat{#2}}%
  \expandafter\def\csname s#1\endcsname{{#2}}%
  \expandafter\def\csname idx#1\endcsname{{\@idxhumbleformat{#2}}}%
}%
\def\humble#1{\@humbleformat{#1}}%
\def\idxhumble#1{\@idxhumbleformat{#1}}%
\makeatother%

% Convenient indexing for humble abbreviations
\def\humbleindex#1#2{\index{#1@\idxhumble{#1}}}


% TODO: Clean up duplicates
\declareHumble{ANOVA}{ANOVA}
\declareHumble{ARD}{ARD}
\declareHumble{BIC}{BIC}
\declareHumble{BMC}{BMC}
\declareHumble{bq}{BQ}
\declareHumble{CRP}{CRP}
\declareHumble{dirpro}{DP}
\declareHumble{HDMR}{HDMR}
\declareHumble{GAM}{GAM}
\declareHumble{GEM}{GEM}
\declareHumble{GMM}{GMM}
\declareHumble{gplvm}{GP-LVM}
\declareHumble{gpml}{GPML}
\declareHumble{GPML}{GPML}
\declareHumble{gprn}{GPRN}
\declareHumble{gpt}{GP}
\declareHumble{gp}{GP}
\declareHumble{HKL}{HKL}
\declareHumble{HMC}{HMC}
\declareHumble{ibp}{IBP}
\declareHumble{iGMM}{iGMM}
\declareHumble{iwmm}{iWMM}
\declareHumble{kCP}{CP}
\declareHumble{kCW}{CW}
\declareHumble{kC}{C}
\declareHumble{KDE}{KDE}
\declareHumble{kLin}{Lin}
\declareHumble{KPCA}{KPCA}
\declareHumble{kPer}{Per}
\declareHumble{kZMPer}{ZMPer}
\declareHumble{kRQ}{RQ}
\declareHumble{kSE}{SE}
\declareHumble{kWN}{WN}
\declareHumble{Lin}{Lin}
\declareHumble{LBFGS}{L-BFGS}
\declareHumble{LIBSVM}{LIBSVM}
\declareHumble{MAP}{MAP}
\declareHumble{mcmc}{MCMC}
\declareHumble{MKL}{MKL}
\declareHumble{MLP}{MLP}
\declareHumble{MNIST}{MNIST}
\declareHumble{MSE}{MSE}
\declareHumble{OU}{OU}
\declareHumble{Per}{Per}
\declareHumble{ZMPer}{ZMPer}
\declareHumble{RBF}{RBF}
\declareHumble{RMSE}{RMSE}
\declareHumble{RQ}{RQ}
\declareHumble{SBQ}{SBQ}
\declareHumble{seard}{SE-ARD}
\declareHumble{sefull}{SE-\textnormal{full}}
\declareHumble{SEGP}{SE-GP}
\declareHumble{SE}{SE}
\declareHumble{SNR}{SNR}
\declareHumble{SSANOVA}{SS-ANOVA}
\declareHumble{SVM}{SVM}
\declareHumble{UCI}{UCI}
\declareHumble{UMIST}{UMIST}
\declareHumble{vbgplvm}{VB GP-LVM}

% Math

\def\graphspace{\mathbf{G}}
\def\Uniform{\mbox{\rm Uniform}}
\def\Bernoulli{\mbox{\rm Bernoulli}}
\def\simiid{\sim_{\mbox{\tiny iid}}}
\def\eqdist{\stackrel{\mbox{\tiny d}}{=}}
\def\ahfunction{\theta}       
\def\AHfunction{\Theta}           % A-H random function
\def\AHvar{U}                     % A-H uniform variables
\def\AHvaralt{V}                  % A-H uniform variables - for bipartite data
\def\larray{W}                    % latent array sampled with A-H
\def\latentspace{\mathbf{W}}      % range of entries
\def\darray{X}                    % data array
\def\dataspace{\mathbf{X}}        % sample space
\def\cfspace{\mathbf{C}}          % space of continuous functions
\def\likelihood{P}

\def\CovData{C}
\def\CovDataAlt{D}

\def\rweightedG{H}
\def\rgraph{G}
\def\SGinf{\mathbb{S}_{\infty}}

\newcommand{\eqd}{\overset{\,_{\!d}}{=}}
\newcommand{\defn}[1]{\emph{#1}}

\newcommand{\Law}{\mathcal{L}}

\newcommand{\NonNegInts}{\mathbb{Z}_+}
\newcommand{\Nats}{\mathbb{N}}
\newcommand{\Rationals}{\mathbb{Q}}
\newcommand{\Reals}{\mathbb{R}}

\newcommand{\as}{\textrm{a.s.}}

\def\[#1\]{\begin{align}#1\end{align}}
\newcommand{\defas}{:=}

\newcommand{\kernel}{\kappa}
\newcommand{\scalefactor}{s}
\newcommand{\lengthscale}{\ell}
\newcommand{\targets}{T}
\newcommand{\noise}{\sigma_\targets}
\newcommand{\pseudopoints}{\eta}
\newcommand{\inputpoints}{\xi}
\newcommand{\covhyppar}{\psi}
\newcommand{\logistic}{\sigma}

\newcommand{\CompOrder}{\mathcal{O}}

%\newcommand{\gp}{{\acro{gp}}}
\let\oldemptyset\emptyset
\newcommand{\kSig}{\boldsymbol\sigma}

\def\subexpr{{\cal S}}
\def\baseker{{\cal B}}
\def\numWinners{k}

\newcommand{\Normal}{\mathcal{N}}
\newcommand{\N}[3]{\mathcal{N}\!\left(#1|#2,#3\right)}
\newcommand{\Nt}[2]{\mathcal{N}\!\left(#1,#2\right)}
\newcommand{\bN}[3]{\mathcal{N}\big(#1|#2,#3\big)}
\newcommand{\boldN}[3]{\text{\textbf{\mathcal{N}}}\big(#1;#2,#3\big)}
\newcommand{\ones}[1]{\mat{1}_{#1}}
\newcommand{\eye}[1]{\mat{E}_{#1}}
\newcommand{\tra}{{^\ensuremath{\mathsf{T}}}}
\newcommand{\trace}{\operatorname{tr}}
\newcommand{\deq}{:=}

\newcommand{\inv}{^{{\mathsmaller{-1}}}}

\newcommand{\layerindex}{\ell}

\newcommand{\procedurename}{ABCD}
\newcommand{\genText}[1]{{\sf #1}}

\newcommand{\asdf}{$^{\textnormal{th}}$}

\newcommand{\binarysum}{\sum_{\bf{x} \in \{0,1\}^D}}
\newcommand{\expect}{\mathbb{E}}
\newcommand{\expectargs}[2]{\mathbb{E}_{#1} \left[ {#2} \right]}
\newcommand{\var}{\mathbb{V}}
\newcommand{\varianceargs}[2]{\mathbb{V}_{#1} \left[ {#2} \right]}
\newcommand{\cov}{\operatorname{cov}}
\newcommand{\Cov}{\operatorname{Cov}}
\newcommand{\covargs}[2]{\Cov \left[ {#1}, {#2} \right]}
\newcommand{\variance}{\mathbb{V}}
\newcommand{\vecop}[1]{\operatorname{vec} \left( {#1} \right)}

\newcommand{\covarianceargs}[2]{\Cov_{#1} \left[ {#2} \right]}
\newcommand{\colvec}[2]{\left[ \begin{array}{c} {#1} \\ {#2} \end{array} \right]}
\newcommand{\tbtmat}[4]{\left[ \begin{array}{cc} {#1} & {#2} \\ {#3} & {#4} \end{array} \right]}

\newcommand{\acro}[1]{{\humble{#1}}}
%\newcommand{\vect}[1]{\boldsymbol{#1}}
\newcommand{\vect}[1]{{\bf{#1}}}
\newcommand{\mat}[1]{\mathbf{#1}}
\newcommand{\pderiv}[2]{\frac{\partial #1}{\partial #2}}
\newcommand{\npderiv}[2]{\nicefrac{\partial #1}{\partial #2}}

\newcommand{\pha}{^{\phantom{:}}}

\newcommand{\argmin}{\operatornamewithlimits{argmin}}
\newcommand{\argmax}{\operatornamewithlimits{argmax}}

% The following designed for probabilities with long arguments

\newcommand{\Prob}[2]{P\!\left(\,#1\;\middle\vert\;#2\,\right)}
\newcommand{\ProbF}[3]{P\!\left(\,#1\!=\!#2\;\middle\vert\;#3\,\right)}
\newcommand{\p}[2]{p\!\left(#1\middle\vert#2\right)}
\newcommand{\po}[1]{p\!\left(#1\right)}
\newcommand{\pF}[3]{p\!\left(\,#1\!=\!#2\;\middle\vert\;#3\,\right)} 
\newcommand{\mean}[2]{{m}\!\left(#1\middle\vert#2\right)}


\newcommand{\valpha}{\boldsymbol{\alpha}}
\newcommand{\va}{\vect{a}}
\newcommand{\vA}{\vect{A}}
\newcommand{\vB}{\mat{B}}
\newcommand{\vb}{\vect{b}}
\newcommand{\vC}{\mat{C}}
\newcommand{\vc}{\vect{c}}
\newcommand{\vecf}{\boldsymbol{f}}
\newcommand{\vell}{\vect{\ell}}
\newcommand{\vepsilon}{\boldsymbol{\epsilon}}
\newcommand{\veps}{\boldsymbol{\epsilon}}
\newcommand{\ve}{\boldsymbol{\epsilon}}
\newcommand{\vf}{\vecf}
\newcommand{\vg}{\vect{g}}
\newcommand{\vh}{\vect{h}}
\newcommand{\vI}{\mat{I}}
\newcommand{\vK}{\mat{K}}
\newcommand{\vk}{\vect{k}}
\newcommand{\vL}{\mat{L}}
\newcommand{\vl}{\vect{l}}
\newcommand{\vmu}{{\boldsymbol{\mu}}}
\newcommand{\vone}{\vect{1}}
\newcommand{\vphi}{{\boldsymbol{\phi}}}
\newcommand{\vpi}{{\boldsymbol{\pi}}}
\newcommand{\vq}{\vect{q}}
\newcommand{\vR}{\mat{R}}
\newcommand{\vr}{\vect{r}}
\newcommand{\vsigma}{{\boldsymbol{\sigma}}}
\newcommand{\vSigma}{\mat{\Sigma}}
\newcommand{\vS}{\mat{S}}
\newcommand{\vs}{\vect{s}}
\newcommand{\vtheta}{{\boldsymbol{\theta}}}
\newcommand{\vu}{\vect{u}}
\newcommand{\vV}{\mat{V}}
\newcommand{\vW}{\mat{W}}
\newcommand{\vw}{\vect{w}}
\newcommand{\vX}{\mat{X}}
\newcommand{\vx}{\vect{x}}
\newcommand{\vY}{\mat{Y}}
\newcommand{\vy}{\vect{y}}
\newcommand{\vzero}{\vect{0}}
\newcommand{\vZ}{\mat{Z}}
\newcommand{\vz}{\vect{z}}

\newcommand{\dist}{\ \sim\ }
\def\given{\,|\,}

% Table stuff
\newcolumntype{C}[1]{>{\centering\let\newline\\\arraybackslash\hspace{0pt}}m{#1}}
\newcolumntype{L}[1]{>{\raggedright\let\newline\\\arraybackslash\hspace{0pt}}m{#1}}
\newcolumntype{R}[1]{>{\raggedleft\let\newline\\\arraybackslash\hspace{0pt}}m{#1}}

\newcommand{\defeq}{\mathrel{:\mkern-0.25mu=}}


%\def\simiid{\sim_{\mbox{\tiny iid}}}
\def\simiid{\overset{\mbox{\tiny iid}}{\sim}}
\def\simind{\overset{\mbox{\tiny \textnormal{ind}}}{\sim}}
\def\eqdist{\stackrel{\mbox{\tiny d}}{=}}
%\newcommand{\distas}[1]{\mathbin{\overset{#1}{\kern \z@ \sim}}}
%TODO: fix this - it worked outside the thesis!
\newcommand{\distas}[1]{\mathbin{\overset{#1}{\sim}}}

\def\Reals{\mathbb{R}}

\def\Uniform{\mbox{\rm Uniform}}
\def\Bernoulli{\mbox{\rm Bernoulli}}
\def\GP{\mathcal{GP}}
\def\GPLVM{\mathcal{GP-LVM}}

% Kernel stuff

\def\inputVar{x}
\def\InputVar{X}
\def\InputSpace{\mathcal{X}}
\def\outputVar{y}
\def\OutputSpace{\mathcal{Y}}
\def\function{f}
\def\kernel{k}
\def\KernelMatrix{K}
\def\SumKernel{\sum}
\def\ProductKernel{\prod}
\def\expression{e}
\def\feat{\vh}

\newcommand{\kerntimes}{ \! \times \!}
\newcommand{\kernplus}{ \, + \,}

% Proof stuff
\theoremstyle{plain}
\newtheorem{theorem}{Theorem}[section]
\newtheorem{lemma}[theorem]{Lemma}
\newtheorem{prop}[theorem]{Proposition}
\newtheorem{proposition}{Proposition}

\newtheorem{thm}{Theorem}%[section]
\newtheorem{lem}[thm]{Lemma}
\newtheorem{cor}{Corollary}

\theoremstyle{definition}
\newtheorem{definition}[thm]{Definition}%[section]
\newtheorem{conj}{Conjecture}[section]
\newtheorem{exmp}{Example}[section]
\newtheorem{example}{Example}[section]
\newtheorem{rem}[thm]{Remark}

\theoremstyle{remark}
%\newtheorem{rem}{Remark}
\newtheorem{note}{Note}
\newtheorem{case}{Case}


\numberwithin{equation}{section}
\numberwithin{thm}{section}

\def\SE{\acro{SE}}
\def\Per{\acro{Per}}
\def\RQ{\acro{RQ}}
\def\Lin{\acro{Lin}}

% Other Aldous Hoover stuff
\def\rweightedG{H}
\def\rgraph{G}
\def\SGinf{\mathbb{S}_{\infty}}